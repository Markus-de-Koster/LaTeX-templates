\documentclass[a4paper,11pt]{article}%Schriftgröße
%\usepackage[sfdefault,thin]{roboto} % font
\usepackage{helvet} % let's use a more readable font for presentations
\renewcommand{\familydefault}{\sfdefault} % belongs to font command above
\usepackage[T1]{fontenc}  
\usepackage[utf8]{inputenc}
\usepackage[ngerman]{babel}%Veröffentlichungssprache
\usepackage{graphicx} 
\graphicspath{ {graphics/} }
\RequirePackage{tikz}
\usepackage{ragged2e}
\usepackage[format=plain,justification=RaggedRight,singlelinecheck=false,font={small},labelsep=space]{caption}
\usepackage{xcolor}	
\usepackage[a4paper]{geometry}
\geometry{left=0cm,right=1.0cm,top=1cm,bottom=1cm,includefoot}%Seitenränder 
%includefoot includes footer in geometry, see: https://ctan.space-pro.be/tex-archive/macros/latex/contrib/geometry/geometry.pdf
\usepackage{svg}
\usepackage{soul} % letter spacing \so or overwrite \sodef<cmd>{font inter-letter-space}{inner space}{outer space}
\sodef\sonamefont{}{.2em}{1em}{0em} %custom spacing for titles / headings
\usepackage[onehalfspacing]{setspace}%Zeilenabstand
%\renewcommand{\\}{\vspace*{0.5\baselineskip} \newline}
\usepackage[hidelinks,breaklinks=true]{hyperref} % links
%\usepackage{breakurl} % mostly for splitting email address
% make resume more machine readable (for ATS)
\input{glyphtounicode}
\pdfgentounicode=1


\hypersetup{pdfauthor={Markus de Koster},%
            pdftitle={Lebenslauf Markus de Koster},%
            pdfsubject={Lebenslauf},%
            pdfkeywords={Lebenslauf, CV, Curriculum Vitae, Cover Letter},%
            pdfproducer={LaTeX},%
            pdfcreator={pdfLaTeX}
}

\pagenumbering{Roman}

%--------%
% Colors %
%--------%
\definecolor{light-gray}{RGB}{153, 153, 153}
\definecolor{medium-gray}{RGB}{109, 109, 109}
\definecolor{dark-gray}{RGB}{64, 64, 64}

\definecolor{light-blue}{RGB}{130, 156, 208}
\definecolor{medium-blue}{RGB}{19,153,198}
\definecolor{dark-blue}{RGB}{32, 54, 143}

%-------%
% Fonts %
%-------%
\newcommand{\mynamefont}[1]{
	{ % limit the scope of this command
		\fontfamily{cmss}\selectfont % select different font
		\fontsize{20}{12}\selectfont % increase font size
		\color{medium-gray}
		\scalebox{1}[1.15]{ % stretch vertically	
			\so{\MakeUppercase{#1}} % this must be in one line (WITHOUT SPACING! see:https://cs.brown.edu/about/system/managed/latex/doc/soul.pdf p8)
		}%
	}
}

\newcommand{\myheadingfont}[1]{
	{ % limit the scope of this command
		\fontsize{12}{12}\selectfont % increase font size
		\color{dark-blue}
		\sonamefont{\MakeUppercase{#1}}
	}
}

\newcommand{\myline}[1]{
	\begin{tikzpicture}
		\fill[color=gray] (0,0) rectangle(#1*\textwidth,0.02);
	\end{tikzpicture}
	\\
}

% Skill bars as a bar chart
% @param 1: name
% @param 2: percentage (e.g. "50.0" for 50%)
\newcommand{\myskillbar}[2]{
	\begin{tikzpicture}
		\node [anchor=west] at (.1,.8) {#1};
		\draw [fill=light-gray] (0,0) rectangle (3,.5);
		\draw [fill={rgb:red,1;green,2;blue,3}] (0,0) rectangle (3/100*#2,.5);
	\end{tikzpicture}
}

% Skill points.
% @param 1: skillpoints out of 5
% @param 2: size of the circles (e.g. 3pt)
\newcommand{\myskillpoints}[2]{
	\begin{tikzpicture}
		\draw (0, 0) circle (#2);
		\draw (0.5, 0) circle (#2);
		\draw (1, 0) circle (#2);
		\draw (1.5, 0) circle (#2);
		\draw (2, 0) circle (#2);
		% a bit hacky, no switch case implemented yet
		\ifnum#1>0
		\filldraw (0, 0) circle (#2);
		\fi
		\ifnum#1>1
		\filldraw (0.5, 0) circle (#2);
		\fi
		\ifnum#1>2
		\filldraw (1, 0) circle (#2);
		\fi
		\ifnum#1>3
		\filldraw (1.5, 0) circle (#2);
		\fi
		\ifnum#1>4
		\filldraw (2, 0) circle (#2);
		\fi
	\end{tikzpicture}
}
\newsavebox\thesmashminipageleft
\newenvironment{smashminipageleft}
  {\begin{lrbox}{\thesmashminipageleft}\begin{minipage}[t]{0.3\textwidth}\vspace{0pt}}
	% see: https://tex.stackexchange.com/a/500284 for why we use vspace here
  {\end{minipage}\end{lrbox}\smash{\usebox{\thesmashminipageleft}}}

\newsavebox\thesmashminipageright
\newenvironment{smashminipageright}
  {\begin{lrbox}{\thesmashminipageright}\begin{minipage}[t]{0.65\textwidth}\vspace{30pt}}
	% see: https://tex.stackexchange.com/a/500284 for why we use vspace here
  {\end{minipage}\end{lrbox}\smash{\usebox{\thesmashminipageright}}\clearpage}


% Picture
\newcommand{\mypic}{
	\begin{tikzpicture}
		\clip (0,0) circle (2cm) node {
			\includegraphics[width=4.5cm]{mypic.jpg}
		};
	\end{tikzpicture}\\
	\color{medium-gray}
	\bigbreak
}
% Contact Data
\newcommand{\mycontact}{
	\href{mailto:my@mail.com}{\includesvg[width=1cm]{icon_mail_circle_2.svg}}\\
	\smallbreak
	\href{mailto:my@mail.com}{my@mail.com}\\
	\bigbreak
	\href{tel:004915770000001}{\includesvg[width=1cm]{icon_phone_circle.svg}}\\
	\smallbreak
	\href{tel:004915770000001}{(+49) 157- 70000001}\\
	\bigbreak
	\includesvg[width=1cm]{icon_location_circle.svg}\\
	\smallbreak
	Mystr. 02\\
	\de{50667 Köln\\}
	\en{50667 Cologne\\}
	\en{Germany\\}
	\bigbreak
	\bigbreak
}

%-----------%
% Languages %
%-----------%

\newif\ifen
\newif\ifde

% Switch this on/off to set the language
%\entrue
\detrue

\newcommand{\en}[1]{\ifen#1\fi}
\newcommand{\de}[1]{\ifde#1\fi}

\newcommand{\cv}{%
  \en{Curriculum Vitae}%
  \de{Lebenslauf}%
}
\newcommand{\coverletter}{%
  \en{Cover Letter}%
  \de{Anschreiben}%
}

% Month abbreviations
\newcommand{\march}{%
  \en{March}%
  \de{März}%
}
\newcommand{\may}{%
  \en{May}%
  \de{Mai}%
}
\newcommand{\october}{%
  \en{Oct.}%
  \de{Okt.}%
}
\newcommand{\december}{%
  \en{Dec.}%
  \de{Dez.}%
}

% Note: if a page is overfilled, a new page will be created before the first one.
% 		Some content may simply not be visible because of this.
% 		Instead manually create a new page
% Important: don't use empty lines between minipage declarations!
\begin{document}
	% ------------- %
%  Left Column  %
% ------------- %
\begin{minipage}[t]{0.3\textwidth}\vspace{0pt}% see: https://tex.stackexchange.com/a/500284 for why we use vspace here
	\begin{center}
		%------------%
		%    Bild    %
		%------------%
		\mypic
		%--------------%
		% Kontaktdaten %
		%--------------%
        \mycontact
	\end{center}
	\vfill
\end{minipage} %do not leave empty lines before the next minipage lines with only comments are okay
%
%\hfill
%\begin{minipage}[t]{0.05\textwidth}\vspace{0pt}%
%	\vfill
%\end{minipage}
\hfill
% -------------- %
%  Right Column  %
% -------------- %
\begin{minipage}[t]{0.65\textwidth}\vspace{28pt}%
	\mynamefont{Markus de Koster}\\ % TODO: change font / (make thinner)
	\hspace*{-3.92em}{\myline{0.9}} % move this line to the left until it almost hits the picture
							% this makes the minipages feel connected
	\color{medium-gray}
	\normalsize \MakeUppercase{\so{\coverletter}} \\
	\\
    An:\\
    Max Mustermann\\
    Serious GmbH\\
    Seriousstr. 05\\
    50667 Köln\\
    
    \bigbreak
    \myheadingfont{Bewerbung zum Software-Entwickler}\\
    \myline{1}
    \renewcommand{\\}{\vspace*{0.8\baselineskip} \newline}
    Sehr geehrter Herr Mustermann,\\
    Lorem ipsum dolor sit amet, consectetur adipiscing elit, 
	sed do eiusmod tempor incididunt ut labore et dolore magna aliqua. 
	Ut enim ad minim veniam, 
	quis nostrud exercitation ullamco laboris nisi ut aliquip ex ea commodo consequat. 
	Duis aute irure dolor in reprehenderit in voluptate velit esse cillum dolore eu fugiat nulla pariatur. 
	Excepteur sint occaecat cupidatat non proident, 
	sunt in culpa qui officia deserunt mollit anim id est laborum\\
    Mit freundlichen Grüßen
    \bigbreak
	\includegraphics[width=4.5cm]{signature.png}\\ %sign manually if printed
    Markus de Koster
\end{minipage}
	\begin{singlespace} 
		% ------------- %
%  Left Column  %
% ------------- %
\begin{smashminipageleft}
	\begin{center}
		%------------%
		%    Bild    %
		%------------%
		\mypic
		%--------------%
		% Kontaktdaten %
		%--------------%
		\mycontact
		%------------%
		% Kenntnisse %
		%------------%
		\myheadingfont{\de{Kenntnisse}\en{Skills}}\\
		\myline{0.5}
		\smallbreak
		\textbf{\de{Sprachen}\en{Languages}}\\
		Java, C, C++\\
		Python \\
		Typescript, SQL \\
		\bigbreak
		\textbf{DevOps}\\
		Git, CI / CD \\
		Docker, Gradle \\
		\bigbreak
		\textbf{Web}\\
		Angular, Django
		\bigbreak
		\textbf{\de{Weiteres}\en{Miscellaneous}}\\
		Linux\\
		embedded systems\\
		\LaTeX \\
		\bigbreak
		\bigbreak
	\end{center}
	\vfill
	%do not leave empty lines before the next minipage lines with only comments are okay
\end{smashminipageleft}
%
%\hfill
%\begin{minipage}[t]{0.05\textwidth}\vspace{0pt}%
%	\vfill
%\end{minipage}
\hfill
% -------------- %
%  Right Column  %
% -------------- %
\begin{smashminipageright}
	\mynamefont{Markus de Koster}\\ % TODO: change font / (make thinner)
	\hspace*{-3.92em}{\myline{0.9}} % move this line to the left until it almost hits the picture
							% this makes the minipages feel connected
	\color{medium-gray}
	\normalsize \MakeUppercase{\so{\cv}} \\
	\\
	

	% sections %
	% -------- %
	%  Arbeit  %
	% -------- %
	\myheadingfont{Beruflicher Werdegang}\\
	\myline{1}
	\textbf{CV Company \null\hfill Nov. 2020 – \december{}  2022} \\
	Abteilung Pretty \LaTeX, Köln \\
	Wissenschaftliche Hilfskraft \null\hfill \may{} 2010 – \december{} 2023 \\
	Bachelorand \null\hfill Feb. 2005 – \may{} 2010 \\ 
	Praktikant \null\hfill Nov. 2000 – Feb. 2005 \\
	\smallbreak
	Lorem ipsum dolor sit amet, consectetur adipiscing elit, 
	sed do eiusmod tempor incididunt ut labore et dolore magna aliqua. 
	Ut enim ad minim veniam, 
	quis nostrud exercitation ullamco laboris nisi ut aliquip ex ea commodo consequat.
	\begin{itemize}
		\itemsep0em %no spacing between items
  		\item Duis aute irure dolor in reprehenderit in voluptate velit esse cillum dolore eu fugiat nulla pariatur. 
    	\item Excepteur sint occaecat cupidatat non proident, 
		\item Sunt in culpa qui officia deserunt mollit anim id est laborum
	\end{itemize}
	% -------- %
	%   Uni    %
	% -------- %
	\bigbreak
	\bigbreak
	\myheadingfont{Akademischer Werdegang}\\
	\myline{1}
	% Master Technische Informatik
	\textbf{Technische Informatik (Master) \null\hfill Sept. 2005 – aktuell }\\
	Master of Science voraussichtlich Mai 2030\\
	Technische Hochschule Köln \\
	\smallbreak
	% Bachelor Technische Informatik
	\textbf{Technische Informatik (Bachelor)\null\hfill Okt. 2004 – Mai 2005 }\\
	\textbf{Bachelor of Science} \\
	Technische Hochschule Köln
	\begin{itemize}
		\itemsep0em %no spacing between items
		\item Schwerpunkt Latein
		\item Abschlussnote 1,0
		\item Bachelorarbeit "Sunt in culpa qui officia deserunt mollit anim id est laborum“ (Note 1,0)
	\end{itemize}	
	Zusätzlich erworbene Zertifikate
	\begin{itemize}
		\itemsep0em %no spacing between items
		\item \LaTeX
		\item Zertifikat Lateinkompetenz
	\end{itemize}
	\smallbreak
\end{smashminipageright}
		% ------------- %
%  Left Column  %
% ------------- %
\begin{smashminipageleft}
	\begin{center}
		%------------%
		%    Bild    %
		%------------%
		\mypic
		%--------------%
		% Kontaktdaten %
		%--------------%
		\mycontact
		%------------%
		%  Sprachen  %
		%------------%
		\myheadingfont{\de{Sprachen}\en{Languages}}
		\myline{0.5}
		\smallbreak
		\textbf{\de{Deutsch}\en{German}}\\
		\smallbreak
		\de{Muttersprache}\en{native}\\
		\bigbreak
		\textbf{\de{Englisch}\en{English}} \\
		\smallbreak
		\de{Fließend (C2)}\en{fluent (C2)}\\
	\end{center}
	\vfill
\end{smashminipageleft}
%
%\hfill
%\begin{minipage}[t]{0.05\textwidth}\vspace{0pt}%
%	\vfill
%\end{minipage}
\hfill
% -------------- %
%  Right Column  %
% -------------- %
\begin{smashminipageright}
	\mynamefont{Markus de Koster}\\
	\hspace*{-3.92em}{\myline{0.9}} % move this line to the left until it almost hits the picture
							% this makes the minipages feel connected
	\color{medium-gray}
	\normalsize \MakeUppercase{\so{\cv}} \\
	\\
	% -------- %
	% Ehrenamt %
	% -------- %
	\bigbreak
	\myheadingfont{Ehrenamtliches Engagement}\\
	\myline{1}
	\textbf{Nachhilfe \hfill 2000 – Nov. 2022 }\\
	Nachhilfeunterricht für Schüler*innen
	\begin{itemize}
		\itemsep0em %no spacing between items
		\item Mathematik
		\item Deutsch
		\item Latein
	\end{itemize}
	
	% -------- %
	% Interessen %
	% -------- %
	\bigbreak
	\bigbreak
	\myheadingfont{Interessen}\\
	\myline{1}
	Klettern, Kochen, Wandern
\end{smashminipageright}
	\end{singlespace}
\end{document}